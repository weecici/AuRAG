\title{Template báo cáo UIT}
\setlength {\marginparwidth }{2cm}

\documentclass[12pt]{article}
\usepackage[T5]{fontenc}
\usepackage[utf8]{inputenc}
\usepackage[vietnamese,english]{babel}
\usepackage{amsmath}
\usepackage{graphicx}
\usepackage[colorinlistoftodos]{todonotes}
\usepackage{listings}
\usepackage{hyperref}
\usepackage{geometry}
\hypersetup{
  colorlinks=true,
  linkcolor=blue,
  filecolor=magenta,
  urlcolor=cyan,
}

\newgeometry{left=3cm, right=3cm}

\begin{document}

\begin{titlepage}

  \newcommand{\HRule}{\rule{\linewidth}{0.5mm}}

  \center

  \textsc{\LARGE Đại học quốc gia TP.HCM}
  \newline
  \textsc{\LARGE Trường đại học công nghệ thông tin}\\[1.5cm]

  \graphicspath{ {./img/} }
  \includegraphics[scale=0.5]{uit.png}\\[1.5cm]
  \textsc{\Large Khoa Khoa học máy tính}\\[0.5cm]
  \textsc{\large Môn học: Các kĩ thuật học sâu và ứng dụng - CS431.Q11 }\\[1.0cm]

  \HRule \\[0.4cm]
  { \huge \bfseries BÁO CÁO ĐỒ ÁN - LẬP NHÓM}\\[0.4cm]
  \HRule \\[2.5cm]

  \begin{minipage}{0.6\textwidth}
    \begin{flushleft} \large
      \emph{Sinh viên thực hiện:}\\
      1. Nguyễn Chí Cường -- 23520199 \\
      2. Lê Thành Thắng Đạt -- 23520251 \\
      3. Vũ Gia Khang -- 23520713 \\
      4. Nguyễn Đăng Khoa -- 23520746 \\
    \end{flushleft}
  \end{minipage}
  ~
  \begin{minipage}{0.35\textwidth}
    \begin{flushright} \large
      \emph{Giảng viên:} \\
      TS. Nguyễn Vinh Tiệp
    \end{flushright}
  \end{minipage}\\[2cm]

  \vfill

\end{titlepage}

\tableofcontents
\newpage

\section{Danh sách thành viên nhóm}
\begin{table}[h!]
  \begin{center}
    \begin{tabular}{ |c| c| c| }
      \hline
      Họ và tên & MSSV & Vai trò \\
      \hline
      Nguyễn Chí Cường & 23520199 & Trưởng nhóm \\
      \hline
      Lê Thành Thắng Đạt & 23520251 & Thành viên \\
      \hline
      Vũ Gia Khang$^{(*)}$ & 23520713 & Thành viên \\
      \hline
      Nguyễn Đăng Khoa & 23520746 & Thành viên \\
      \hline
    \end{tabular}
  \end{center}
\end{table}
$^{(*)}$ Bạn Vũ Gia Khang đã email cho giảng viên về vấn đề làm nhóm 4 người.

\section{Đề tài}

Trong bối cảnh công nghệ trí tuệ nhân tạo (AI) phát triển mạnh mẽ, các mô hình tạo sinh (Generative Models) như \textbf{Stable Diffusion} hay \textbf{Midjourney} đã mang lại khả năng tạo ra hình ảnh chất lượng cao chỉ từ văn bản mô tả. Nhưng những mô hình AI này thường \textbf{khó giữ nguyên đặc trưng khuôn mặt người dùng và mất nhiều thời gian huấn luyện}. InstantID cho phép tạo ảnh cá nhân hóa \textbf{nhanh, chính xác và không cần fine-tuning}, giúp \textbf{tối ưu trải nghiệm người dùng} trong các ứng dụng như tạo avatar, nghệ thuật số hay nền tảng mạng xã hội. Vì vậy, nhóm chọn đề tài này để nghiên cứu và ứng dụng công nghệ \textbf{tạo sinh giữ danh tính trong thời gian thực}.

\section{Phân công công việc}
\begin{figure}[h!]
  \centering
  \includegraphics[width=15cm]{./img/pipeline.png}
  \caption{Pipeline dự kiến của hệ thống}
\end{figure}
Các node quan trọng trong pipeline gồm:
\begin{itemize}
  \item ID Embedding (Face Embedding): Ảnh khuôn mặt đầu vào được đưa qua một mô hình nhận diện khuôn mặt (ví dụ antelopev2 hoặc InsightFace) để trích xuất vector đặc trưng (embedding). Vector này chứa thông tin nhận dạng duy nhất của người đó.
  \item Image Adapter (Lightweight Adapted Module): Một mạng nông (lightweight adapter) chuyển embedding khuôn mặt thành tín hiệu hướng dẫn (visual prompt) cho mô hình khuếch tán. Mục tiêu là giúp mô hình “hiểu” đặc trưng khuôn mặt mà không cần fine-tuning.
  \item IdentityNet (ID controller): Module chuyên biệt dùng để đảm bảo bảo toàn danh tính. Nó kết hợp embedding khuôn mặt và tín hiệu từ prompt để điều khiển các tầng attention của mô hình khuếch tán, giữ ổn định đặc trưng khuôn mặt trong suốt quá trình sinh ảnh.
  \item Base model: Dựa trên mô hình SD1.5 hoặc SDXL, thực hiện quá trình khuếch tán ngược (denoising) để tạo ra hình ảnh hoàn chỉnh, vừa đúng với mô tả văn bản, vừa giữ nguyên danh tính người trong ảnh gốc.
\end{itemize}

Bảng phân công công việc:

\begin{table}[h!]
  \begin{center}
    \begin{tabular}{ |c| c| c| }
      \hline
      Họ và tên & MSSV & Công việc  \\
      \hline
      Nguyễn Chí Cường & 23520199 & Face Encoder (ID Embedding) \\
      \hline
      Lê Thành Thắng Đạt & 23520251 & Image Adapter (Visual Prompt) \\
      \hline
      Vũ Gia Khang & 23520713 & IdentityNet (ID Controller) \\
      \hline
      Nguyễn Đăng Khoa & 23520746 & Base Model (Stable Diffusion) + Kết hợp Pipeline \\
      \hline
    \end{tabular}
  \end{center}
\end{table}

\section{Kết quả dự kiến đạt được}
Những kết quả dự kiến nhóm sẽ đạt được bao gồm:
\begin{itemize}
  \item Một demo về hệ thống của nhóm với ảnh được lấy từ nguồn thispersondoesnotexist
    \begin{itemize}
      \item \textbf{Metric cho bài toán sinh ảnh:}
        \begin{itemize}
          \item \textbf{Identity Similarity (ID Score):} Cosine similarity giữa embedding khuôn mặt gốc và khuôn mặt trong ảnh sinh ra (đo bằng ArcFace).
          \item \textbf{Image Quality:} FID (Fréchet Inception Distance) hoặc CLIP Score để đánh giá độ chân thực và sự ăn khớp với prompt..
        \end{itemize}
    \end{itemize}

  \item Một báo cáo chi tiết về quá trình thực hiện dự án, so sánh các mô hình đã sử dụng, các thử nghiệm và đánh giá mô hình.
\end{itemize}

\end{document}