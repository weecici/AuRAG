\title{Template báo cáo UIT}
\setlength {\marginparwidth }{2cm}

\documentclass[12pt]{article}
\usepackage[T5]{fontenc}
\usepackage[utf8]{inputenc}
\usepackage[vietnamese,english]{babel}
\usepackage{amsmath}
\usepackage{graphicx}
\usepackage[colorinlistoftodos]{todonotes}
\usepackage{listings}
\usepackage{hyperref}
\usepackage{geometry}
\hypersetup{
  colorlinks=true,
  linkcolor=blue,
  filecolor=magenta,
  urlcolor=cyan,
}

\newgeometry{left=3cm, right=3cm}

\begin{document}

\begin{titlepage}

  \newcommand{\HRule}{\rule{\linewidth}{0.5mm}}

  \center

  \textsc{\LARGE Đại học quốc gia TP.HCM}
  \newline
  \textsc{\LARGE Trường đại học công nghệ thông tin}\\[1.5cm]

  \graphicspath{ {./img/} }
  \includegraphics[scale=0.5]{uit.png}\\[1.5cm]
  \textsc{\Large Khoa Khoa học máy tính}\\[0.5cm]
  \textsc{\large Môn học: Các kĩ thuật học sâu và ứng dụng - CS431.Q11 }\\[1.0cm]

  \HRule \\[0.4cm]
  { \huge \bfseries BÁO CÁO ĐỒ ÁN - LẬP NHÓM}\\[0.4cm]
  \HRule \\[2.5cm]

  \begin{minipage}{0.6\textwidth}
    \begin{flushleft} \large
      \emph{Sinh viên thực hiện:}\\
      1. Nguyễn Chí Cường -- 23520199 \\
      2. Lê Thành Thắng Đạt -- 23520251 \\
      3. Vũ Gia Khang -- 23520713 \\
      4. Nguyễn Đăng Khoa -- 23520746 \\
    \end{flushleft}
  \end{minipage}
  ~
  \begin{minipage}{0.35\textwidth}
    \begin{flushright} \large
      \emph{Giảng viên:} \\
      TS. Nguyễn Vinh Tiệp
    \end{flushright}
  \end{minipage}\\[2cm]

  \vfill

\end{titlepage}

\tableofcontents
\newpage

\section{Danh sách thành viên nhóm}
\begin{table}[h!]
  \begin{center}
    \begin{tabular}{ |c| c| c| }
      \hline
      Họ và tên & MSSV & Vai trò \\
      \hline
      Nguyễn Chí Cường & 23520199 & Trưởng nhóm \\
      \hline
      Lê Thành Thắng Đạt & 23520251 & Thành viên \\
      \hline
      Vũ Gia Khang$^{(*)}$ & 23520713 & Thành viên \\
      \hline
      Nguyễn Đăng Khoa & 23520746 & Thành viên \\
      \hline
    \end{tabular}
  \end{center}
\end{table}
$^{(*)}$ Bạn Vũ Gia Khang đã email cho giảng viên về vấn đề làm nhóm 4 người.

\section{Đề tài}
Nhiều bạn sinh viên có thói quen không ghi chép và ghi âm bài học, ghi chép và ghi âm không đầy đủ, hay không xem lại bản ghi sau mỗi buổi học dẫn đến việc tiếp thu kiến thức không hiệu quả và hay quên nhưng bài học cũ.

Do đó, để hỗ trợ các bạn sinh viên trong việc thẩm thấu kiến thức đã học, nhóm chúng em sẽ xây dựng:  \textbf{"Hệ thống tóm tắt và trả lời câu hỏi bài học từ những file âm thanh được ghi lại trong các buổi học"}.

\section{Phân công công việc}
\begin{figure}[h!]
  \centering
  \includegraphics[width=15cm]{./img/pipeline.png}
  \caption{Pipeline dự kiến của hệ thống}
\end{figure}
Các node quan trọng trong pipeline gồm:
\begin{itemize}
  \item Speech Recognize (Speech to Text): Chuyển đổi file âm thanh thành văn bản.
  \item Text Summarize: Tóm tắt văn bản thu được từ bước Speech Recognize.
  \item Vectorize: Chuyển đổi văn bản thành các vector biểu diễn.
  \item Retrieve: Truy xuất các văn bản liên quan từ cơ sở dữ liệu vector dựa trên câu hỏi đầu vào.
  \item RAG (Answer Generation): Sử dụng một LLM để trả lời câu hỏi dựa trên kết quả trả về của bước truy xuất các văn bản liên quan được lưu trong cơ sở dữ liệu vector.
\end{itemize}

Bảng phân công công việc:

\begin{table}[h!]
  \begin{center}
    \begin{tabular}{ |c| c| c| }
      \hline
      Họ và tên & MSSV & Công việc  \\
      \hline
      Nguyễn Chí Cường & 23520199 & Retrieve + RAG \\
      \hline
      Lê Thành Thắng Đạt & 23520251 & Vectorize + Text Summarize \\
      \hline
      Vũ Gia Khang & 23520713 & Vectorize + Text Summarize \\
      \hline
      Nguyễn Đăng Khoa & 23520746 & Speech Recognize \\
      \hline
    \end{tabular}
  \end{center}
\end{table}

\section{Kết quả dự kiến đạt được}
Những kết quả dự kiến nhóm sẽ đạt được bao gồm:
\begin{itemize}
  \item Một demo về hệ thống của nhóm với audio được trích xuất từ các video bài giảng của môn học \textbf{CS431} trên moodle và youtube.
  \item Bảng đánh giá hiệu quả của hệ thống (so sánh giữa các tổ hợp mô hình khác nhau).
    \begin{itemize}
      \item Baseline: Whisper (Speech Recognize), GPT-OSS-120B (Summarization), EmbeddingGemma-300M (Vectorize), Hybrid Search (Retrieve), GPT-OSS-120B (RAG).
      \item Metric cho task Q\&A: Độ chính xác (Accuracy) của các câu trả lời so với đáp án đúng (dataset là tổng hợp các câu hỏi quiz trong khóa học CS431 trên moodle).
      \item Metric cho task Text Summarization: BLEU score, BertScore giữa văn bản đã được tóm tắt và văn bản gốc.
    \end{itemize}
  \item Một báo cáo chi tiết về quá trình thực hiện dự án, so sánh các mô hình đã sử dụng, các thử nghiệm và đánh giá mô hình.
\end{itemize}

\end{document}